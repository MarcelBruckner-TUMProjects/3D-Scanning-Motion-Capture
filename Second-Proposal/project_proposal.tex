% pdflatex project_proposal.tex && bibtex project_proposal.aux && pdflatex project_proposal.tex && open project_proposal.pdf
\documentclass[a4paper,pagesize 10pt]{scrartcl}

\usepackage{graphicx}
%\usepackage{scalefnt}
%\usepackage{textfit}

\begin{document}


\begin{center}{\Huge\textbf{Project Proposal}}\end{center}
\begin{center}{\Large\textbf{Body Animation}}\end{center}

\section{Abstract}

%
%Write a short abstract of your planned project.
%
%Cite papers that you want to use as references (e.g. Adelson et al.~\cite{adelson1984pyramid}).
%
%Include an overview figure that shows your planned processing pipeline.
%

The goal of this project is to implement the famous german game "Torwandschie\ss en where the player has to kick a football through two circular openings on an erected goal with a limited amount of tries. 
The team which has the most successful shots wins.
We will realize this by using a kinect and its skeleton tracking SDK \cite{kinect_basic} to map a player and its movements to a virtual model \cite{source0}.
The program will detect the motion of kicking, then calculate a force and apply it to a virtual ball which will then be shot at a virtual "Torwand". 
The visualization and physics simulation will be realized in Unity \cite{unity}

% TODO: add image
%\begin{figure}[h]
%	\centering
%	\includegraphics[width=\linewidth]{overview}
%	\caption{Method overview.}
%	\label{fig:overview}
%\end{figure}

\section{Requirements}
\begin{itemize}	
	\item Kinect sensor
\end{itemize}

\section{Team}
\begin{itemize}
\item Marcel Bruckner
\item Kevin Bein
\item Jonas Schulz
\item Chandramohan Sudar
\end{itemize}


% references
{\small
	\bibliographystyle{plain}
	\bibliography{project_proposal}
}

\end{document}


